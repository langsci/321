\chapter{Acknowledgements}
This book is a significantly revised version of my PhD thesis \citep{lemke2020}. Writing the thesis, conducting the research reported here and turning it into this book would not have been possible without the help, support and advice of a lot of people whom I'd like to thank in what follows. 

First of all, there are my supervisors Ingo Reich, Heiner Drenhaus and Oliver Bott. Without Ingo this book would not exist, not only because he started project B3 in the \emph{Sonderforschungsbereich} (SFB, Collaborative Research Center) 1102, in whose context the research reported here was conducted, but also because of his way of supervising my PhD thesis. I am grateful to him for always being there for discussing methodological and theoretical issues and for his valuable comments on earlier versions of the manuscript. I would like to thank Heiner, my second supervisor, for the extremely helpful discussions and suggestions on theoretical issues, statistical analysis and the presentation of my results as well as comments on previous versions of the manuscript. I am very grateful to Oliver, my external supervisor, for his comments both in the review and during my PhD defense. 

I would also like to thank the other members of my doctoral committee, Julia Knopf, Elke Teich and Stefan Thater, for their time and their questions during my PhD defense. The research reported in this book has been funded by the Deutsche Forschungsgemeinschaft (DFG, German Research Foundation).\footnote{Gefördert durch die Deutsche Forschungsgemeinschaft (DFG) – Projektnummer 232722074 – SFB 1102.}

When I began thinking about the possibility of working on a PhD thesis, I was certain that I wanted to do this a collaborative environment. Fortunately, in Saarbrücken I found that, both within SFB 1102 and the Department of Modern German Linguistics. I would like to thank everybody with whom I discussed aspects of this work at the joint colloquium of the German and English studies, the SFB's PhD days and of course at the conferences where I could present my research thanks to the DFG's generous funding. 

Some colleagues in Saarbrücken helped me out with more specific issues: I would like to thank Philipp Rauth for always sharing his expertise in generative syntax, Julia Stark for drawing the visual stimuli used in experiment \ref{exp:case-production} and allowing me to publish them in this book and Simon Ostermann for sharing the language model file for pre-processing DeScript. I am particularly indebted to Lisa Schäfer, with whom I shared the office, conference travel and the corresponding leisure, hikes, food and beers during most of the time that I worked on my thesis. She also contributed to preprocessing the DeScript corpus and spent a lot of time proof-reading and commenting on previous versions of this work. I would also like to thank the (former) student assistants Luise Ehrmantraut, Fabian Ehrmantraut, Jonathan Watkins, Natascha Kraushaar and Matt Kuhn for helping with the annotation of production data, the construction of experimental materials and -- in the case of Matt -- proofreading large parts of the thesis.

At this point, I would like to acknowledge my professors and lecturers at Freie Universität Berlin and Humboldt Universität zu Berlin, in particular Guido Mensching, Anke Lüdeling, Manfred Krifka, Anton Benz and Hubert Truckenbrodt (in order of appearance), who raised my interest in linguistics, convinced me of the importance of doing empirical and quantitative research, and/or allowed me to conduct my first experiments and corpus studies.

The people mentioned so far supported me in one way or another in writing my thesis or even thinking about doing so. For allowing me to publish it in a way that makes the results of my research publicly accessible, I thank the Open Germanic Linguistics series editors, Michael T. Putnam, Laura Catharine Smith and Richard Page and everybody at Language Science Press. In particular, I am grateful to Stefan Müller, Sebastian Nordhoff and Felix Kopecky for their support during the preparation of the final manuscript, the Language Science Press community proof-readers for their time and John T. Hale for providing a very detailed and helpful review on the manuscript.

Schließlich möchte ich meiner Familie und meinen Freund*innen für morali\-sche und logistische Unterstützung in dieser Zeit danken, insbesondere Madaida, Jan, Heike und Julia für zeitaufwändiges Katzenhüten und -taxifahrten. Mei\-nen Eltern Heike, Matthias, Andreas und Carmen und Großeltern Edith und Fritz danke ich dafür, mich während meinem Studium und davor immer unterstützt und zu haben.